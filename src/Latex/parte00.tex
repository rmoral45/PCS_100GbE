%%%%%%%%%%%%%%%%%%%%%%%%%%%%%%%%%%%%%%%%%%%%%%%%%%%%%%%%%%%%%%%%%%% 
%% Section
%%%%%%%%%%%%%%%%%%%%%%%%%%%%%%%%%%%%%%%%%%%%%%%%%%%%%%%%%%%%%%%%%%%
\section{Procesadores embebidos y unidades aritméticas en FPGAs}
\begin{frame}
  \frametitle{\textbf{Tabla de Contenidos}}
  \begin{center}
    {\vspace{-1.5cm}\Large \textbf{Sección \thesection}\vspace{0.5cm}}
    
    \begin{beamercolorbox}[
      sep=8pt,center]{part title}
      \usebeamerfont{part title}
      \textbf{\insertsection}
    \end{beamercolorbox}
  \end{center}
\end{frame}


\begin{frame}%[allowframebreaks]
  \frametitle{\textbf{Procesadores embebidos y unidades aritméticas en FPGAs}}
  %\framesubtitle{<En caso que haga falta se agrega un subtitulo >}
  \begin{block}{\textbf{Introducción}}
    \begin{itemize} \justifying\footnotesize
    \item Las FPGAs han surgido como dispositivos que permiten desarrollar
      aplicaciones de procesamiento de señales de alto rendimiento.
		\item En este campo de aplicación, los FPGA han superado a la tradicional
      tecnología de procesadores de señal \(DSP\). Sin importar cuantos MACs pueda
      colocar el proveedor de DSP en un chip, esto no compite contra los cientos de
      estas unidades que pueden ser colocadas en un dispositivo FPGA de alto
      rendimiento.
		\item En la actualidad, los FPGA cuentan con procesadores incorporados,
      interfaces estándar y bloques de procesamiento de señales constituidos por
      multiplicadores, sumadores, registros y multiplexores. Diferentes
      dispositivos de una misma familia poseen un gran número de estas unidades
      embebidas, y con el tiempo se espera que el número de las mismas en un
      mismo dispositivo siga creciendo.
    \end{itemize}
  \end{block}
\end{frame}

\begin{frame}%[allowframebreaks]
  \frametitle{\textbf{Procesadores embebidos y unidades aritméticas en FPGAs}}
  %\framesubtitle{Filtro IIR}
  \vspace{-0.3cm}
  \begin{figure}[!t] \centering
    \includegraphics[scale=0.5]{./figures/block.eps}
    \caption*{Bloques DSP en FPGAs Xilinx.}
  \end{figure}
\end{frame}

\begin{frame}%[allowframebreaks]
  \frametitle{\textbf{Instanciación de bloques}}
  \framesubtitle{Comparación entre dos tecnologías}
  \vspace{-0.3cm}
  \begin{figure}[!t] \centering
    \includegraphics[scale=0.26]{./figures/block.eps}
    \caption*{\centering \footnotesize Reporte de Síntesis - Spartan3}
  \end{figure}
  \vspace{-0.5cm}
  \begin{figure}[!t] \centering
				\includegraphics[scale=0.26]{./figures/block.eps}
				\caption*{\centering \footnotesize Reporte de Síntesis - Virtex4.}
		\end{figure}
\end{frame}

\begin{frame}%[allowframebreaks]
  \frametitle{\textbf{Sumadores}}
  \framesubtitle{Sumadores Básicos}
  \begin{block}{\textbf{Half Adders}}
    \begin{itemize} \justifying\footnotesize
    \item Un half adder \textit{(HA)} es un circuito combinacional usado para
      sumar dos bits, $a_{i}$ y $b_{i}$, sin un acarreo de entrada. La suma
      $s_{i}$ y el acarreo de salida $c_{i}$ están dados por:
    \end{itemize}
    \begin{eqnarray}
      s_{i} &=& a_{i} \oplus b_{i} \\
      c_{i} &=& a_{i}b_{i}
    \end{eqnarray}
    \begin{itemize} \justifying\footnotesize
    \item El path crítico posee una latencia de 1, y corresponde a la longitud
      de cualquiera de los dos paths \textit{(de cualquiera de las dos
        operaciones)}.
    \end{itemize}
  \end{block}
\end{frame}


\begin{frame}%[allowframebreaks]
  \frametitle{\textbf{Los Cuatro Niveles de Abstracción}}
  \framesubtitle{Nivel Comportamental}

  \begin{exampleblock}{\textbf{Asignación Bloqueante}}
    \vspace{-0.5cm}
    \begin{tabular}[c]{lr}
      %\hspace{-0.4cm}
      \begin{minipage}[t]{0.45\linewidth}
        \vspace{0.5cm}
        \begin{itemize} \justifying\footnotesize
        \item El path crítico posee una latencia de 1, y corresponde a la longitud
          de cualquiera de los dos paths \textit{(de cualquiera de las dos
            operaciones)}.
        \end{itemize}
      \end{minipage}
      \hspace{0.3cm}
      \begin{minipage}[t]{0.45\linewidth}
        \vspace{0.5cm}
        \begin{figure}[!t] \centering
          \includegraphics[scale=0.8]{./figures/block.eps}
        \end{figure}
      \end{minipage}
    \end{tabular}
  \end{exampleblock}

\end{frame}

\begin{frame}%[allowframebreaks]
  \frametitle{\textbf{Los Cuatro Niveles de Abstracción}}
  \framesubtitle{Nivel Comportamental}
  %\vspace{-0.5cm}
  \begin{tabular}[c]{lr}
    % \hspace{-0.4cm}
    \begin{minipage}[t]{0.45\linewidth}
      \begin{exampleblock}{\textbf{Asignación Bloqueante}}
        % \vspace{0.5cm}
        \begin{itemize} \justifying\footnotesize
        \item El path crítico posee una latencia de 1, y corresponde a la longitud
          de cualquiera de los dos paths \textit{(de cualquiera de las dos
            operaciones)}.
        \end{itemize}
      \end{exampleblock}
      \end{minipage}
      \hspace{0.3cm}
      \begin{minipage}[t]{0.45\linewidth}
        \vspace{0.5cm}
        \begin{figure}[!t] \centering
          \includegraphics[scale=0.8]{./figures/block.eps}
        \end{figure}
      \end{minipage}
    \end{tabular}
\end{frame}